%###############################################################################
% Plantilla de LaTeX para la carátula de los DT

% Adaptada de la carátula de DT que estaba en formato 
% Word

% Elaborada por: Juan Ignacio Urruty

%###############################################################################

%###############################################################################

% Preambulo - paquetes necesarios para que 'funcione'
\documentclass[12pt]{article}

% Opcional - tipo de fuente para documento y fórmulas
% Idealmente, deberían dejar todo con las mismas fuentes
% Si usan otra fuente que no sea Times New Roman
% Deberian cambiar esta parte
\usepackage{times}
\usepackage{mathptmx}

% En caso de que sea trabajo en español
\usepackage[spanish]{babel}

% Paquetes que ayudan con las tablas e imágenes
\usepackage{subfigure}
\usepackage{graphicx}
\usepackage{float}
\usepackage[labelfont=bf]{caption}
\usepackage[shortlabels]{enumitem}


% Paquete necesario para crear los colores
\usepackage[svgnames]{xcolor}

% En estas líneas se crean los colores que se usan en la carátula
% el gris y el rojo
\definecolor{mygrey}{RGB}{127, 127, 127}
\definecolor{myred}{RGB}{164, 44, 52}

% Paquete para poder cambiar los márgenes
% Los márgenes establecidos por las pautas son de 2.5cm
% No modificar
\usepackage[a4paper]{geometry}
\geometry{top=2.5cm, bottom=2.5cm, left=2.5cm, right=2.5cm}

% Esto es para modificar el interlineado.
\usepackage{setspace}
\onehalfspacing

% Este paquete es para que aparezcan las líneas punteadas en la carátula
% en la parte de ISSN
\usepackage[dash,dot]{dashundergaps}

% Este paquete es para poder crear cuadros como el rojo de la carátula
% Es de mucha utilidad
\usepackage{mdframed}


%###############################################################################

%###############################################################################

\begin{document}

% Esto es para que no se le ponga número a estas páginas y que las mismas no 
% aparezcan luego en el índice
\pagestyle{empty}
\pagenumbering{gobble}

\onehalfspacing

%###############################################################################
% CARÁTULA
%###############################################################################

\noindent\makebox[\textwidth]{\includegraphics[scale=0.52]{logos/Logo FCEA - IECON - UDELAR.jpg}}


\vspace{7cm}


% Esto crea un cuadro de texto con el título del trabajo
\begin{mdframed}[backgroundcolor=white, linecolor=white, innerleftmargin=4.5cm, innerrightmargin=0.5cm]

% Tamaño de la fuente del titulo
\large

% Título del trabajo
\textcolor{mygrey}{Una evaluación del impacto de la plantilla de \LaTeX. Una comparación con la plantilla de Word}
    
\end{mdframed}

% Esta línea saca el espacio que quedaba entre los objetos
\vspace{-0.8cm}

% Esta línea de código traza la línea negra
\rule{15cm}{0.5pt}

% Esta línea de código crea el cuadro de texto de los nombres
% autores
\begin{mdframed}[backgroundcolor=white, linecolor=white, innerleftmargin=4cm, innerrightmargin=0.5cm]

% Tamaño de la fuente del nombre de los autores
\normalsize

% Esta línea de código es para quitarle la indentación
% a los nombres y que se alineara mejor con lo de arriba
\noindent

\vspace{-0.5cm}

% Reemplazar con los nombres y apellidos
Nombre y apellido autora 1

Nombre y apellido autora 2

Nombre y apellido autora 3

\end{mdframed} 

% Agrega espacio
\vspace{2cm}

% Esto crea el cuadro de texto con fondo rojo
\begin{mdframed}[backgroundcolor=myred, linecolor=myred, leftmargin=-40pt, rightmargin=-40pt]

% En esta parte habría que reemplazar por la fecha %
\hspace{5cm} \textcolor{white}{INSTITUTO DE ECONOMÍA} \hfill \textcolor{white}{Julio 2023} % Acá habría que cambiar    
\end{mdframed}

\vspace{-0.7cm}


% Esto crea el cuadro de texto de serie de documentos
\begin{mdframed}[backgroundcolor=white, linecolor=white, leftmargin=-40pt, rightmargin=-40pt]

\small

% Acá habría que cambiar el número
\hspace{5cm} Serie Documentos de Trabajo \hfill DT 05/23 % En esta parte

\end{mdframed}

\vspace{4cm}

\begin{flushright}

ISSN: \dashuline{\, \textcolor{mygrey}{1510-9305}  \, } \quad \quad (en papel)


ISSN: \dashuline{ \, \textcolor{mygrey}{1688-5090}  \, } \quad \quad (en línea)
    
\end{flushright}    


% Salto de página. Dejar este comando
\newpage


%###############################################################################
% AGRADECIMIENTOS
%###############################################################################


% Esta línea es para que el texto comience más abajo
\null \vspace{7cm}

% Comentar en el main en caso de no querer escribir agradecimientos. En otro 
% caso, redactar la dedicatoria luego del comando \small
\noindent \small  Personas, instituciones u organizaciones que brindaron su apoyo en la elaboración del trabajo o en la formación del autor.

\vspace{1cm}

% Cambiar toda la información después del \small 
\noindent \small Forma de citación sugerida para este documento: Apellido, N., Apellido, N., y Apellido, N. (Año). Una evaluación del impacto de la plantilla de \LaTeX. Una comparación con la plantilla de Word. Serie Documentos de Trabajo, DT 99/2099. Instituto de Economía, Facultad de Ciencias Económica y Administración, Universidad de la República, Uruguay.


\newpage


%###############################################################################
% RESUMEN Y ABSTRACT
%###############################################################################

% Esa linea es para que las notas al pie aparezcan con un asterisco
\renewcommand{\thefootnote}{\fnsymbol{footnote}}

\begin{center}

\large Una evaluación del impacto de la plantilla de \LaTeX. Una comparación con la plantilla de Word

\vspace{7mm}

\small Nombre y apellido autora 1\footnote[1]{Institucion, Email}, Nombre y apellido autora 2\footnote[2]{Institución, Email} y Nombre apellido autora 3\footnote[3]{Institución, Email}

\end{center}

\vspace{0.5cm}


% Esta linea es para que el comando \begin{abstract} no ponga otro título
\renewcommand{\abstractname}{Resumen}
\begin{abstract}



\footnotesize  El resumen debe seguir el estilo que generalmente se encuentra en los papers académicos, donde se mencionan los objetivos, metodologías y resultados principales del trabajo. Este apartado debe dar una idea rápida al lector acerca de lo que trata el trabajo, para que este pueda decidir si se ajusta a sus intereses para seguir leyendo.

\vspace{3mm}

Palabras clave: 

\vspace{3mm}

Clasificación JEL:

\end{abstract}

\vspace{0.5cm}

\renewcommand{\abstractname}{Abstract}
\begin{abstract}



\footnotesize  El resumen debe seguir el estilo que generalmente se encuentra en los papers académicos, donde se mencionan los objetivos, metodologías y resultados principales del trabajo. Este apartado debe dar una idea rápida al lector acerca de lo que trata el trabajo, para que este pueda decidir si se ajusta a sus intereses para seguir leyendo.

\vspace{3mm}

Keywords:

\vspace{3mm}

JEL classification: 

\end{abstract} 


\newpage


\end{document}
